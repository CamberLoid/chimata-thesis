\chapter{绪论}

\section{选题背景与研究意义}
近年来,在电子商务高速发展和网络技术不断进化的大环境下,移动支付这一新兴的概念,在互联网金融业务和移动网络交叉融合下产生,凭借其移动性、便携性和及时性,颠覆了传统的基于银行和纸币的交易方式,给当下大众的支付消费习惯带来巨大影响。\cite{mobilepaymentrisk}

在一个典型的移动支付的交易场景中,交易双方首先在线下商量好交易金额,然后通过某一支付服务提供方进行转账操作,在此过程中不需要用户进行现金的交易,款项的转移在支付服务提供方的云服务器中完成,也可能涉及到和银行方的交互。

由于移动支付服务依托于云计算,其也会受到一般云计算服务可能遇到的风险,包括内部人员监守自盗,或是外部攻击人员成功获取到密钥后解密加密过的信息\cite{lastpass}。一旦交易信息泄露,攻击者可以轻易推断用户的各种敏感信息,例如:支付能力、信用情况和消费习惯等。因此,对交易金额等敏感数据的恰当处理,在官方监管机构的要求下,已经成为如今支付服务的必须安全标准之一。\cite{gov_pay}

% 研究意义
本文着手于在用户交易过程中,通过同态加密等密码学工具,确保用户的交易金额只由交易双方知道而不泄露给第三方。于此核心目的之上,通过 R-LWE 的重加密方法\cite{Mouchet2020LattigoAM,lattigoRepo}实现了交易的正确性,也使用了 CKKS 支持对浮点数进行同态运算的特性为交易手续费的计算留下了空间。基于 CKKS 方案对用户交易数据隐私保护的研究,有助于后人拓展和提出更完备,同时也具有更强功能性的交易隐私保护方案,也有助于提高用户对移动支付的信任度。

\section{国内外研究现状}

目前对基于同态加密的用户交易隐私保护的研究的概括可以分解为:同态加密研究、基于同态加密的隐私保护研究、用户交易隐私保护研究等。本节将对这些研究现状进行分别阐述。

\subsection{同态加密方案的研究现状}

同态加密(Homomorphic Encryption, HE)指的是一类允许对密文进行特定运算的算法,由 Rivest 等人在 1978 年首次提出\cite{rivest1978data},而全同态加密(Fully Homomorphic Encryption, FHE)指的是允许对密文进行任意次同态运算的一类同态加密算法。

在 FHE 被提出以前,就有许多经典加密系统具有部分同态加密(Partial HE)性质,如最为流行的 RSA 加密系统,以及 Paillier 等。因为有成熟的硬件实现,到目前已有许多基于上述加密方法的部分同态性质的隐私保护模型被提出。2009年,斯坦福大学的博士研究生 Craig Gentry 在他的博士学位论文中,提出了第一个全同态加密方案\cite{homenc}。Craig 的论文标志着第一代全同态加密方案的出现。方案设计了一个允许进行有限次数的同态密文加法和乘法的类同态加密(Somewhat FHE)算法,并通过自举(Bootstrapping)以将有限的层级同态加密拓展至全同态加密。在 Craig 之后越来越多的全同态加密方案被提出,其中包括了 BGV 方案,将明文空间拓展至全部实数域乃至虚数的 CKKS\cite{cryptoeprint:2016/421} 方案等。

另一方面,随着全同态加密的关注度不断增加,越来越多的全同态加密库被发布,如微软的 SEAL\cite{sealcrypto}、IBM 的 HELib,以及由开源社区维护的 OpenFHE\cite{OpenFHE} 和 Lattigo\cite{Mouchet2020LattigoAM} 等。这些全同态加密密码学库的发布标志着全同态加密技术正在从理论阶段向应用阶段进行转换\cite{ZQL-SEAL},对推动学者进行相关领域实际应用的研究有着重大实际意义。

\subsection{基于同态加密的隐私保护研究}

因为全同态加密方案和部分同态加密方案的允许对密文进行同态运算的重要特性,使得它们在隐私保护领域有着广泛的应用,包括隐私保护的机器学习、区块链上数据的隐私保护、安全医疗和安全投票等。

例如,2018 年,一个结合了区块链和同态加密的电子医疗记录隐私保护方案被提出\cite{Homo_Medi},实现了保险公司等第三方在无法获取客户明文医疗记录的情况下,仍然可以判断是否理赔的功能。

另一方面,同态加密也被应用在联邦学习中,用来保护用户数据不被不可信参与者攻击。\cite{FL_HE}在联邦学习中,多方联合训练模型一般需要交换中间结果,如果直接发送明文的结果可能会有隐私泄露风险。在这种场景下,同态加密就可以发挥很重要的作用。

\subsection{用户交易隐私保护研究现状}

对用户交易信息的隐私保护是一项重要的研究课题。在现有的研究中,对用户交易隐私保护的研究往往结合了同态加密、区块链和零知识证明等相关密码学工具,以使用户敏感信息只对交易双方知晓,同时也允许了部分或全部的交易信息可以被监管方审计。

2020 年,姜轶涵等人在他的文献里\cite{ACT}运用了同态加密、零知识证明和数字签名等密码学手段,提出了一个密态可审计交易方案。结合这些手段,该方案保证了交易数据隐私和正确性,同时也满足了可审计性。2022 年,学者周星光提出了另一个面向航空差旅消费的可监管的交易隐私保护方案。\cite{ZXG_Air_privacy}该方案在 Boneh 等人基于身份的公钥加密方案上进行了拓展,并利用双公钥方案和零知识证明方案,实现了高效的监管方和交易双方之间的核算,在引入了额外接收方的前提下依然能够高效运行。

\section{本文主要研究目的和内容}

本文基于全同态加密方案 CKKS\cite{cryptoeprint:2016/421} 和 ECDSA 数字签名方案,设计了一个具备交易金额隐私保护的支付方案,保证交易金额的隐私性和正确性。本文也基于该方案,提出了一个使用 Golang 编写的简单库实现。该库在客户端方面实现了对交易的加密和解密,对交易信息的签名等;在服务端库中实现了对交易数据的验证,密钥转换和同态账户的余额更新等操作。此外亦配套了一个简单的服务端实现。

本文主要研究内容为:

\begin{enumerate}
    \item 对现有的交易隐私保护方法和全同态加密算法进行了简要的研究,了解了目前全同态加密算法的局限性;对所使用算法所基于的困难问题 R-LWE 亦进行了简要的调研。
    \item 根据方案的安全需求和功能需求,确定了方案的技术路线;使用 Golang 进行基本函数的代码实现,并使用 SQLite3 进行用户和交易数据的存储,使用 http 协议进行客户端和服务端之间的通信。
    \item 研究了方案依赖的全同态加密密码学库 Lattigo\cite{lattigoRepo},了解了 Lattigo 的 CKKS 子模块和 RLWE 子模块,及模块内暴露\footnote{又称“被导出(Exported)”或“公有的 (Public)”,在软件工程领域中指外部可以访问的函数,和私有的(Private)相对,后者不可以由外部程序调用而只能被库内的方法调用}的方法,为后续的方案实现做准备。
    \item 编写了一个简单的库实现,并附有集成的单元测试。
\end{enumerate}

\section{论文结构安排}

本文共分为六章,具体内容如下:

第一章:绪论。该章节介绍了本文的研究背景和意义,介绍了目前全同态加密的研究进展,以及其当前在隐私保护领域的应用,以及本文的主要研究内容和结构安排。

第二章:基础知识。该章节包括了对本文提出的方案所使用的技术路线、密码学工具等的介绍,包括全同态加密、环上的误差还原问题、CKKS 方案及使用的密码学库 Lattigo 的相关介绍进行阐述。

第三章:交易金额隐私保护的支付方案设计。这一章主要包括了方案的需求分析、构造、隐私保护能力和安全性分析,其中方案构造部分阐述了方案的基本假设和交易各阶段。

第四章:代码层级的具体设计以及实现。本章主要包括对方案的具体实现,包括客户端和服务端的具体实现的描述,以及对相关代码的描述和解析。

第五章:实验评估。本章利用了 Golang 工具链集成的单元测试工具,对方案的部分模块进行了测试和性能评估。

第六章:总结和展望。对本文的成果和不足进行总结和展望。
