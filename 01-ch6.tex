\chapter{结论与展望}

随着互联网技术、电子商务高速发展和网络技术不断进化的大环境下,移动支付得到前所未有的普及,移动支付服务的安全性也成为了一个重要的问题。然而没有一个系统是完全安全的,我们可以注意到,数据泄露事件不断发生,常有服务商将用户的数据进行明文储存而后泄露,也有服务商将用户的敏感数据进行交易。这些风险都是不可忽视的。因此,本文基于 CKKS 方案,利用其允许对浮点数进行同态加密的特性,提出了一个简单的对用户交易金额进行保护的交易方案。本文取得的主要成果如下:

\begin{enumerate}
    \item 分析了方案的安全需求等;
    \item 在安全需求之上,提出了交易方案的抽象构造;
    \item 研究了同态加密库 Lattigo 的使用方法和特性,了解了其对 CKKS 方案的相关暴露的方法等。在此基础之上,编写了交易方案的大部分实现;
    \item 研究了 Golang 工具链的测试框架,并在此之上编写了代码实现的部分测试用例;
    \item 进行测试,验证了方案的可行性。总的来说,代码使用的 Lattigo 提供的加解密方法和同态计算方法有着较好的性能,且加解密精度符合要求
\end{enumerate}

然而,由于时间有限,本文所提出的方案和配套的实现还有诸多不足和遗憾:

\begin{enumerate}
    \item 未能做到基于私钥-公钥的代理重加密\cite{proxy_re_encryption}。在撰写本文正在进行收尾工作时,笔者发现了有基于 CKKS 方案的代理重加密方案被提出\cite{cryptoeprint:2017/410}。然而目前笔者未找到该方案的 Golang 实现,也暂时无如此的精力将该方案进行 Golang 的移植。若能实现该方案,则可以进一步保证本交易方案的安全性。
    \item 作为简单的交易方案,并未引入零知识证明和区块链等技术。
    \item 方案中描述了监管方,但是样例实现并未给出监管方的具体实现。这意味着本方案实现并不能保证用户出现透支时,监管方能够及时地对用户进行透支的处理。
\end{enumerate}
