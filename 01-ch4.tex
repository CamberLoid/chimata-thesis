\chapter{交易方案的详细设计和代码实现}

\section{引言}

本章主要对本文提出的交易金额隐私保护的交易方案的样例实现进行描述,方案的实现设计抽象如下图所示。

% 插入示意图

实现分为多个层级,最底层为程序设计语言工具链 Golang、数据库 SQLite3 以及全同态加密库 Lattigo;往上一层为共用的内部包(Internal Package),包括定义用户的包 \verb|internal/user|、定义共用数据库操作的包 \verb|internal/db|、定义单笔交易基本信息的包 \verb|internal/transaction|,以及定义密钥的包 \verb|internal/db|;再向上一层则为定义了客户端和服务端主要操作的包 \verb|internal/\{client,server\}lib|。最上层则为本文附带的可能的服务端样例实现。此外,本文亦提供了一些集成测试样例,可以通过 Golang 工具链的测试功能进行运行。对测试的具体描述和结果将放置在下一章描述。

本实现使用了以下第三方包:

\begin{itemize}
    \item \verb|uuid|(From google):用于生成和管理 UUID;
    \item \verb|lattigo|:含有 CKKS 方案的密码学库;
    \item \verb|go-sqlite3|:访问 SQLite 数据库文件;
\end{itemize}

\section{方案支持层}

\subsection{数据库}

数据库是本文样例实现中重要的一个基础组成部分。本文选择了 SQLite3 作为客户端和服务端的数据库引擎,因为 SQLite3 是一个轻量级的数据库引擎,它的数据存储在单个文件中,而不需要额外的服务器配置。这使得本样例实现不需要过多的配置。本文的数据库子包定义在 \verb|internal/db| 中。

对于客户端和服务端的数据库,本文使用了相同的数据库结构,但是在实际使用时,客户端和服务端的数据库是分开的。本文包括了以下数个表:

\begin{itemize}
    \item \verb|TABLE Transactions|:记录交易数据的表
    \item \verb|TABLE Users|:记录用户数据的表
    \item \verb|TABLE SwitchingKeys|:记录重加密密钥的表
    \item \verb|TABLE CKKSKeyChains|:记录加密算法 CKKS 方案所使用公私钥的表\footnote{其中私钥非必需存在,可以为 NULL,即无此私钥。ECDSA 同理}
    \item \verb|TABLE ECDSAKeyChains|:记录签名算法 ECDSA 所使用公私钥的表
\end{itemize}

本文的数据库具体设计的抽象如下图所示:

% 插入数据库设计图

在 Golang 中使用下面的语句导入 SQLite3 相关库:

\begin{minted}{go}
import (
    "database/sql" // 标准库

    database "github.com/CamberLoid/Chimata/internal/db" // 方案内部库
    _ "github.com/mattn/go-sqlite3" // 外部第三方库,操作 SQLite3 数据库引擎
)
\end{minted}

由于本文使用的数据库涉及到对外码的使用,使用了外码约束来保证数据库的一致性,因此在初始化数据库时需要额外添加指令:

\begin{minted}{go}
    db.Exec("PRAGMA foreign_keys = ON;")
\end{minted}
 
\subsection{交易信息}

结构体 \verb|Transaction| 包含在子包 \verb|internal/transaction| 内,被上层包所使用。其定义如下文所示

\begin{minted}{go}
type Transaction struct {
    ConfirmingPhase   string    `json:"confirmingPhase"`
    UUID              uuid.UUID `json:"uuid"`
    Sender            uuid.UUID `json:"sender"`
    Receipt           uuid.UUID `json:"receipt"`
    CTSender          []byte    `json:"ctSender"`
    CTReceipt         []byte    `json:"ctReceipt"`
    SigCTSender       []byte    `json:"sigCtSender"`
    CTSenderSignedBy  uuid.UUID `json:"ctSenderSignedBy"`
    SigCTReceipt      []byte    `json:"sigCTReceipt"`
    CTReceiptSignedBy uuid.UUID `json:"ctReceiptSignedBy"`
    TimeStamp         int64     `json:"timestamp"` //unix时间戳
    IsValid           bool      `json:"isValid"`
}    
\end{minted}

\begin{itemize}
    \item \verb|ConfirmingPhase|:交易的确认阶段,取值为 \verb|{"unconfirmed", "confirmed", "failed"}|,分别表示未确认、已确认和无效;
    \item \verb|Sender, Receipt|:作为外码标识交易的转出和转入方,其底层类型为 \verb|[16]byte|;
    \item \verb|CTSender, CTReceipt|:经过 \verb|rlwe.CipherText.MarshalBinary()| 方法反序列化的密文
    \item \verb|SigCTSender, SigCTReceipt|:对密文的签名,由标准库 \verb|crypto/ecdsa| 中 \verb|ecdsa.SignASN1()| 产生;
\end{itemize}

该子包也包含了一些对交易信息的通用的辅助函数:

\begin{itemize}
    \item \verb|func (t Transaction) GetSenderCT() (ct *rlwe.Ciphertext, err error)|:输出序列化后的转出方密文;
    \item \verb|func (t Transaction) GetReceiptCT() (ct *rlwe.Ciphertext, err error)|:输出序列化后的转入方密文;
    \item \verb|func CalcFixedFee(ct *rlwe.Ciphertext, rate float64) (fee *rlwe.Ciphertext)|:计算固定费率的手续费;
    \item \verb|func CalcRatedFee(ct *rlwe.Ciphertext, rate float64) (fee *rlwe.Ciphertext)|:计算按比例费率的手续费;
\end{itemize}

\subsection{其他杂项:密钥链、用户和其他}

\subsubsection*{密钥链}

子包 \verb|internal/key| 定义了密钥链的相关结构体和函数,包括以下部分:

\begin{itemize}
    \item \verb|type CKKSKeyChain struct|:CKKS 方案所使用的密钥链结构体,包含了公私钥对象的指针\verb|CKKSPrivateKey *rlwe.SecretKey, CKKSPublicKey  *rlwe.PublicKey|,以及一个唯一标识符 \verb|Identifier|
    \item \verb|type ECDSAKeyChain struct|:类似 CKKSKeyChain 的定义,此处不再赘述。
    \item \verb|type CKKSPayload interface|:包含了 \verb|MarshalBinary() ([]byte, error)| 和 \verb|UnmarshalBinary([]byte) error| 两个序列化和反序列化函数的接口,对应 lattigo 中的密文对象、公私钥对象和重加密密钥对象。
\end{itemize}

\subsubsection{用户}

子包 \verb|internal/user| 定义了基本的用户类型。代码声明如下:

\begin{minted}{go}
type User struct {
    UserIdentifier    uuid.UUID // 或者是 [16]bytes
    UserName          string
    UserCKKSKeyChain  []key.CKKSKeyChain
    UserECDSAKeyChain []key.ECDSAKeyChain
}
\end{minted}

\subsubsection{其他}

子包 \verb|internal/misc| 定义了一些其他的辅助函数,包括:

\begin{itemize}
    \item \verb|func NewCiphertext() *rlwe.Ciphertext|:创建新的空密文对象;
    \item \verb|func GenerateSwitchingKey(skIn, skOut *rlwe.SecretKey) *rlwe.SwitchingKey|:将 lattigo 中的 \verb|ckks.KeyGenerator.GenSwitchingKey()| 进行封装的函数;
    \item \verb|func CKKSMsgRound(v float64) float64|:将由 CKKS 方案解密的浮点数,进行保留两位小数的四舍五入。
\end{itemize}

\section{客户端}

方案客户端实现以包 \verb|internal/clientlib| 提供,分为以下几个部分:

\begin{itemize}
    \item \verb|client.go|:包含 Client 对象及以其为接收器的高层级函数。\\
    一个 Client 对象包含了主用户的 UUID 和对应的数据库对象的指针。以 Client 为接收器,实现了转账请求函数、认可转账函数,以及获取交易信息中交易金额的函数。
    \item \verb|user.go|:包含了对 User 对象的定义和操作;\\
    此处 User 对象继承了 \verb|internal/user.User|,并实现了包括使用用户主密钥链签名,以及进行解密的函数。
    \item \verb|transaction.go|:以 User 对象为接收器,定义了交易相关的函数,包括了转账请求函数和认可转账函数的中层实现;
    \item \verb|server.go|:包含客户端与服务端的网络通信相关函数的文件;
    \item \verb|db.go|:包含客户端对数据库的初始化操作函数;
    \item \verb|crypto.go|:包含对 CKKS 方案中加密和解密函数的包装;
\end{itemize}

因精力有限,故本文不再附带客户端的 CLI\footnote{Command-Line Interface,命令行界面,常见于 Unix-like 系统中} 实现,仅提供该客户端库的集成测试函数,以进行实验和测试。测试过程和结果见下一章。

\subsection{客户端的主要函数}

本小节将对客户端被导出的主要函数进行简要介绍。

\subsubsection{密码学相关}

\begin{itemize}
    \item \verb|func CKKSEncryptAmount(float64, *rlwe.PublicKey) *rlwe.Ciphertext|:本文对 Lattigo 提供的加密函数的封装,以浮点数金额和公钥对象作为输入,输出密文对象的指针。
    \item \verb|func CKKSDecryptAmountFromCT(*rlwe.Ciphertext, *rlwe.SecretKey) float64 |:上述函数的反函数。
\end{itemize}

\subsubsection{网络通信相关}

\begin{itemize}
    \item \verb|func RegisterUser|:向服务端注册用户基本信息,包括 UUID、用户名、公钥链等;
    \item \verb|func RegisterSwk|:向服务端注册重加密密钥,也包括输入方和输出方的 UUID;
    \item \verb|func (*User) CreateTransferJob|:创建转账请求,方法包括对金额的加密、签名等;
    \item \verb|func (*User) GetBalance|:获取密文余额;
    \item \verb|func GetTransactionFromServer|:获取单笔交易的信息,其中交易金额为密文。
\end{itemize}

此外,在 \verb|server.go| 中还定义了变量 \verb|var ConfigServerURL string = DefaultServerURL|,用于指定服务端的地址。

\section{服务端}

方案服务端实现以包 \verb|internal/serverlib| 和简单的程序实现 \verb|cmd/server| 提供。

\subsection{服务端主程序}

服务端中,其与客户端的通信经由 HTTP(S) 完成。对于每个暴露的 EndPoint,都有一个函数处理其接收的请求。服务端现阶段暴露的 EndPoints 如下:

\begin{itemize}
    \item \verb|/register/user|:注册用户基本信息的接口。若请求不合法则返回状态码 400\footnote{Bad Request},其他错误返回状态码 500\footnote{Internal Server Error,即服务端内部发生错误} 并包含错误描述,否则返回状态码 200 并中断连接;
    \item \verb|/register/swk|:向服务器提交重加密密钥的接口。若请求不合法则返回状态码 400,其他错误返回状态码 500 并包含错误描述,否则返回状态码 200 并中断连接;
    \item \verb|/transaction/create/bySenderPK| 和 \verb|/transaction/create/byReceiptPK|:创建转账请求的接口。当转账请求不合法时返回 HTTP 代码 400,签名验证失败返回状态码 401,其他错误返回状态码 500 并包含错误描述,否则返回状态码 200 并包含交易现阶段的所有信息;
    \item \verb|/transaction/confirm|:接收转账的接口,与上一条类似。
    \item \verb|/transaction/get|:获取交易信息的接口。该接口不需要鉴权。若请求不合法则返回状态码 400,其他错误返回状态码 500 并包含错误描述,否则返回状态码 200 并包含交易现阶段的所有信息;
    \item \verb|/user/GetBalance|:获取密文余额的接口。若请求不合法则返回状态码 400,其他错误返回状态码 500 并包含错误描述,否则返回状态码 200 并包含用户的密文余额;
\end{itemize}

主函数如下:

\begin{minted}{go}
func main(){
    loggerInit()
    http.HandleFunc("/transaction/create/bySenderPK", 
        HandlerTransactionCreateBySenderPK)
    // ... 忽略其他接口声明
    
    // 打开数据库
    if Database, err = InitDatabase(); err != nil {
        CriticalLogger.Fatal(err.Error())
    }
    defer Database.Close()

    // 开始监听
    InfoLogger.Printf("Listening: %v", ConfigListenAddr+":"+ConfigListenPort)
    if err := http.ListenAndServe(ConfigListenAddr+":"+ConfigListenPort, nil); err != nil {
        log.Fatal(err)
    }
}
\end{minted}

\subsection{服务端库 serverlib}

服务端库 \verb|serverlib| 中包含了服务端的主要逻辑,包括对数据库的操作、对交易的处理、对用户的处理等。包括的重点函数和结构体如下:

\begin{itemize}
    \item \verb|func ReEncryptCTWithSwk(ct, swk)|:调用 \verb|evaluator.SwitchKeysNew| 方法,对密文进行重加密。同时也包含处理重加密失败时将 panic 转换为错误 error 的逻辑。
    \item \verb|func KeySwitchSenderToReceipt| 与 \verb|func KeySwitchReceiptToSender|:这两个方法从更高的层级,调用上述 \verb|ReEncryptCTWithSwk()|。
    \item \verb|func ValidateSignatureForCipherText(ct interface{}, sig []byte, pk *ecdsa.PublicKey)|:对密文的签名进行验证,包含处理不同种类密文,即密文对象和/或密文反序列化后的字节流的逻辑。
    \item \verb|func GetUpdatedBalance|:对交易后的账户进行处理的逻辑,包括对密文余额的更新。
    \item \verb|func InitializeNewReceiptPKTransaction|、\verb|func InitializeNewSenderPKTransaction| 和 \verb|func FinishTransaction|:处理交易信息的支援性函数
    \item \verb|type User struct|:该类型继承了 \verb|users.User|,并多一个 Balance 对象。
\end{itemize}
